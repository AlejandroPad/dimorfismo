\documentclass{article}
\usepackage[spanish]{babel}
\usepackage[utf8]{inputenc}
\usepackage{amssymb, amsmath, amsbsy} % simbolitos
\usepackage{mathdots} % para el comando \iddots
\usepackage{csvsimple}
\usepackage{pythontex}
\usepackage{ragged2e}
\usepackage{adjustbox}
\usepackage{chngpage} %tablas fuera de margenes
\usepackage[margin=1.0in]{geometry}
\renewcommand{\baselinestretch}{1.5}

\begin{pycode}
import json
import csv

##Declaro variables a utilizar 
##Primer archivo

data_processed_1 = '../data/processed/best_logistic_model_parameters_laal_ig.json'
with open(data_processed_1, encoding='utf8') as results_1_file:
    best_parameters = json.load(results_1_file)

b_normalization_variables = best_parameters ["normalization_parameters"]
b_minimum_value = b_normalization_variables ["minimum_value"]
b_maximum_value = b_normalization_variables ["maximum_value"]

b_model_variables = best_parameters["model_parameters"]
m_intercept = b_model_variables[0]
b_skull_length = b_model_variables[1]
b_peak_height = b_model_variables[2]
b_peak_length = b_model_variables[3]
b_tarsus = b_model_variables [4]


#tercer archivo
b_model_variables = []
with open('../data/processed/best_models_table.csv') as results_2_file:
    reader = csv.DictReader(results_2_file)
    for row in reader:
        b_model_variables.append(row)

\end{pycode}

\begin{document}

\title{Ecuación Logística}
\author{Andrea Sánchez}
\maketitle
\begin{flushleft}

Teniendo la función logística:
\begin{equation}
    \label{tripe_igualdad_sencilla}
    f(z) = \frac{1}{1+\mbox{\mbox{exp}}\left ( -z\right )}.
\end{equation}

Al ser $z$ una representación del escalar obtenido mediante la suma de multiplicaciones una a una de las componentes cartesianas de los vectores $\vec{\beta}$ y $\vec{N}$, y además, teniendo que $N_{0}=1$, entonces podemos expandir la ecuación (\ref{tripe_igualdad_sencilla}) a la forma:
\begin{equation}
    \label{ecuacion_expandida}
    f(z) = f(\vec{\beta} \cdot \vec{N}) = \frac{1}{1+\mbox{exp}\left (-\sum_{i=0}^{n} \beta _{i}N_{i} \right )} = \frac{1}{1+\mbox{exp}\left [ -\left ( \beta _{0} + \beta _{1}N_{1}+...+\beta_{n}N_{n} \right ) \right ]} ,
\end{equation}

de modo que, podemos reescribir la ecuación (\ref{ecuacion_expandida}) a la forma :
\begin{equation}
    \label{ecuacion_expandida_reescrita}
    \frac{1}{1+\mbox{exp}\left [ -\left ( \beta _{0} + \beta _{1}N_{1}+...+\beta_{n}N_{n} \right ) \right ]} = \frac{1}{1+\mbox{exp}\left [ -\left ( \beta _{0} + \beta _{1}N(x_{1})+...+\beta_{n}N(x_{n})\right ) \right ]} .
\end{equation}

Notemos que en la ecuación (\ref{ecuacion_expandida_reescrita}) la normalización $\vec{N}$ depende de un $x_{i}$. Definimos $A_{i}$ y $B_{i}$ como:

\begin{equation}
    \label{maximos_A}
    A_{i}= \underset{j\in \left \{ 1,...,m \right \}}{\mbox{mín}}\left ( x_{i}^{j} \right ) ,
\end{equation}

\begin{equation}
    \label{minimos_B}
    B_{i}= \underset{j\in \left \{ 1,...,m \right \}}{\mbox{máx}}\left ( x_{i}^{j} \right ) .
\end{equation}

Donde, en este caso, el superíndice $j$ representa el $j$-ésimo individuo del conjunto de $m$ albatros, mientras que el subíndice $i$ simboliza la $i$-ésima anatomía en medición.

Por lo tanto, sea 
\begin{equation} \label{normalizacion} N\left ( x_{i} \right )= \frac{x_{i} - \underset{j\in [m ]}{\mbox{mín}}\left (  x_{i}^{j}\right )}{\underset{j\in [m ]}{\mbox{máx}}\left (  x_{i}^{j}\right )-\underset{j\in [m ]}{\mbox{mín}}\left (  x_{i}^{j}\right )} .
\end{equation}

Sustituyendo (\ref{normalizacion}) en (\ref{ecuacion_expandida_reescrita}) tenemos que: 
\begin{equation} \label{ecuacion_chorizo}
    \frac{1}{1+\mbox{\mbox{exp}}\left ( -z\right )} = \frac{1}{1+\mbox{exp}\left [ -\left ( \beta _{0} + \beta _{1}\frac{x_{1}- \underset{j\in [m]}{\mbox{mín}}\left (x_{1}^{j}\right )}{\underset{j\in [m ]}{\mbox{máx}}\left (x_{1}^{j}\right )-\underset{j\in [m]}{\mbox{mín}}\left (  x_{1}^{j}\right )}+...+\beta_{n}\frac{x_{n} - \underset{j\in [m]}{\mbox{mín}}\left (  x_{n}^{j}\right )}{\underset{j\in [m]}{\mbox{máx}}\left (  x_{n}^{j}\right )-\underset{j\in [m]}{\mbox{mín}}\left (  x_{n}^{j}\right )}\right ) \right ]} .
\end{equation}


\newpage
Con base en una muestra de 135 Albatros de Laysan recolectada en Isla Guadalupe desde 2014 a 2018 podemos generar los siguientes resultados.
Donde, a partir de los ecuación (\ref{maximos_A}) obtenemos los valores maximos y por las ecuación (\ref{minimos_B}) los mínimos de cada variable predictora.
%Agregar valores de .csv

\begin{table}[ht]
    \centering
    \caption{Variables predictoras para principales componentes de análisis.}
    \bigskip
    \begin{adjustbox}{width=1\textwidth}
    \renewcommand{\arraystretch}{1.3}
    \begin{tabular}{|c|c|c|c|c|c|c|}
    \hline
    \textbf{Morfometría} & \textbf{Valor Mínimo} & \textbf{Valor Máximo} & \textbf{Valor Estimado} & \textbf{Error Estándar} & \textbf{Valor Z} & \textbf{Pr($>\left|z\right|$)} \\
    \hline
    (Intercept) & - & - & \py{b_model_variables[0]['(Intercept)']} & \py{b_model_variables[0]['error_std_intercept']} & \py{b_model_variables[0]['valor_z_intercept']} & \py{b_model_variables[0]['pr_intercept']}  \\
\hline
Longitud Craneo & \py{b_model_variables[0]['min_head_length']} & \py{b_model_variables[0]['max_head_length']} & \py{b_model_variables[0]['head_length']} & \py{b_model_variables[0]['error_std_head_length']} & \py{b_model_variables[0]['valor_z_head_length']} & \py{b_model_variables[0]['pr_intercept']}\\
\hline
Altura Pico & \py{b_model_variables[0]['min_alto_pico']} & \py{b_model_variables[0]['max_bill_height']} & \py{b_model_variables[0]['bill_depth']} & \py{b_model_variables[0]['error_std_alto_pico']} & \py{b_model_variables[0]['valor_z_bill_height']} & \py{b_model_variables[0]['pr_alto_pico']}\\
\hline
Longitud Pico & \py{b_model_variables[0]['min_bill_length']} & \py{b_model_variables[0]['max_bill_length']} & \py{b_model_variables[0]['bill_length']} & \py{b_model_variables[0]['error_std_bill_length']} & \py{b_model_variables[0]['valor_z_bill_length']} & \py{b_model_variables[0]['pr_bill_length']}\\
\hline
Tarsus & \py{b_model_variables[0]['min_tarsus']} & \py{b_model_variables[0]['max_tarsus']} & \py{b_model_variables[0]['Tarsus']} & \py{b_model_variables[0]['error_std_tarsus']} & \py{b_model_variables[0]['valor_z_tarsus']} & \py{b_model_variables[0]['pr_tarsus']}\\
\hline
Ancho Craneo & \py{b_model_variables[0]['min_head_width']} &\py{b_model_variables[0]['max_head_width']} &\py{b_model_variables[0]['head_width']} &\py{b_model_variables[0]['error_std_head_width']} &\py{b_model_variables[0]['valor_z_head_width']} & \py{b_model_variables[0]['pr_head_width']} \\
\hline
    \end{tabular}
    \end{adjustbox}
    \\[10pt]
    Resultados obtenidos del archivo \texttt{tabla\_mejores\_modelos.csv}
    \label{modeloLogistico}
\end{table}

%Agrega valores de .json

\begin{table}[h]
    \centering
    \caption{Variables predictoras obtenidas después de la normalizacion según la ecuación (\ref{normalizacion}) para principales componentes de análisis.}
    \bigskip
    \renewcommand{\arraystretch}{1.3}
    \begin{tabular}{|c|c|c|c|}
    \hline
    \textbf{Morfometría} & \textbf{Valor Mínimo} & \textbf{Valor Máximo} & \textbf{Valor Estimado}\\
    \hline
    (Intercept) & - & - & \py{m_intercept["Estimate"]} \\
\hline
Longitud Craneo & \py{b_minimum_value["head_length"][0]} & \py{b_maximum_value["head_length"][0]} & \py{b_head_length["Estimate"]} \\
\hline
Altura Pico & \py{b_minimum_value["bill_depth"][0]} & \py{b_maximum_value["bill_depth"][0]} & \py{b_peak_height["Estimate"]} \\
\hline
Longitud Pico & \py{b_minimum_value["bill_length"][0]} & \py{b_maximum_value["bill_length"][0]}& \py{b_peak_length["Estimate"]} \\
\hline
Ancho Craneo & \py{b_minimum_value["head_width"][0]} & \py{b_maximum_value["head_width"][0]}& \py{b_head_width["Estimate"]} \\
\hline
    \end{tabular}
    \label{mejorModeloLogistico}
    \\[10pt]
    Resultados obtenidos del archivo \texttt{parametros\_mejor\_modelo\_logistico\_laal\_ig.json}
\end{table}

\end{flushleft}
\end{document}
