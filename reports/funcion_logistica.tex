\documentclass{article}
\usepackage{vmargin}
\usepackage{graphicx}
\usepackage[spanish]{babel}
\usepackage[utf8]{inputenc}
\usepackage{amssymb, amsmath, amsbsy} % simbolitos
\usepackage{upgreek} % para poner letras griegas sin cursiva
\usepackage{cancel} % para tachar
\usepackage{mathdots} % para el comando \iddots
\usepackage{mathrsfs} % para formato de letra
\usepackage{stackrel} % para el comando \stackbin
\setpapersize{A4}

\begin{document}
\title{Ecuación Logística}
\maketitle
Teniendo la ecuación logística:
\begin{equation}
    \label{simple_equation}
    \frac{1}{1+e^{-z}} = \frac{1}{1+\mbox{\mbox{exp}}\left ( -z\right )} = \frac{1}{1+\mbox{\mbox{exp}}\left (-\sum_{i=0}^{n} \beta _{i}N_{i} \right )}
\end{equation}

Donde, al tener que $N_{0}=1$, entonces la ecuación (1) puede ser expandida de la forma:
\begin{equation}
    \label{simple_equation}
    \frac{1}{1+\mbox{exp}\left (-\sum_{i=0}^{n} \beta _{i}N_{i} \right )} = \frac{1}{1+\mbox{exp}\left [ -\left ( \beta _{0} + \beta _{1}N_{1}+...+\beta_{n}N_{n} \right ) \right ]}
\end{equation}

Donde (2) puede ser reescrito de la forma:
\begin{equation}
    \label{simple_equiation}    
    \frac{1}{1+\mbox{exp}\left [ -\left ( \beta _{0} + \beta _{1}N_{1}+...+\beta_{n}N_{n} \right ) \right ]} = \frac{1}{1+\mbox{exp}\left [ -\left ( \beta _{0} + \beta _{1}N_{1}(x_{1})+...+\beta_{n}N_{n} (x_{n})\right ) \right ]}
\end{equation}

Notemos que en (3) la normalización $N$ depende de un $X_{i}$ .

Definimos $A_{i}$ y $B_{i}$ como:

\begin{equation}
    \label{simple_equiation}
    A_{i}= \mbox{mín}_{j=\left \{ 1,...,m_{i} \right \}}\left ( x_{i}^{j} \right )
\end{equation}

\begin{equation}
    \label{simple_equiation}
    B_{i}= \mbox{máx}_{j=\left \{ 1,...,m_{i} \right \}}\left ( x_{i}^{j} \right )
\end{equation}

Donde, en este caso, el superíndice $j$ representa la anatomía en medición, mientras que el superíndice $i$ simboliza al i-ésimo albatro del conjunto.


Por lo tanto, sea 
\begin{equation} N_{i} \left ( X_{i} \right )= \frac{x_{i}^{j} - \underset{j\in [m_{i} ]}{\mbox{mín}}\left (  x_{i}^{j}\right )}{\underset{j\in [m_{i} ]}{\mbox{máx}}\left (  x_{i}^{j}\right )-\underset{j\in [m_{i} ]}{\mbox{mín}}\left (  x_{i}^{j}\right )}
\end{equation}

Sustituyendo (6) en (3) tenemos que: 
\begin{equation}
    \frac{1}{1+\mbox{exp}\left [ -\left ( \beta _{0} + \beta _{1}\frac{x_{1}^{j} - \underset{j\in [m_{1} ]}{\mbox{mín}}\left (  x_{1}^{j}\right )}{\underset{j\in [m_{1} ]}{\mbox{máx}}\left (  x_{1}^{j}\right )-\underset{j\in [m_{1} ]}{\mbox{mín}}\left (  x_{1}^{j}\right )}+...+\beta_{n}\frac{x_{n}^{j} - \underset{j\in [m_{n} ]}{\mbox{mín}}\left (  x_{n}^{j}\right )}{\underset{j\in [m_{n} ]}{\mbox{máx}}\left (  x_{n}^{j}\right )-\underset{j\in [m_{n} ]}{\mbox{mín}}\left (  x_{n}^{j}\right )}\right ) \right ]}
\end{equation}
\end{document}